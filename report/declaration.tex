%-------------------------------------------------------------------------------------------------------
\documentclass{article}             		% autres choix : letter, book, report
\usepackage[utf8]{inputenc}         		% encodage du fichier source
\usepackage[T1]{fontenc}            		% gestion des accents (pour les pdf)
\usepackage[francais, english]{babel}       % rajouter éventuellement english
\usepackage{textcomp}              			% caractères additionnels
\usepackage{amsmath,amssymb}       			% pour les maths
\usepackage{lmodern}                		% remplacer éventuellement par txfonts, fourier, etc.
\usepackage[a4paper]{geometry}      		% taille correcte du papier
\usepackage{graphicx}               		% pour inclure des images
    \graphicspath{{images/}}     			% spécifier le dossier des images
\usepackage[table,xcdraw]{xcolor}           % pour gérer les couleurs
\usepackage{svg}
\usepackage{microtype}              		% améliorations typographiques
\usepackage{parskip}
\usepackage{fancyhdr}
\usepackage{amsmath}
%\usepackage{glossaries}
\usepackage[nopostdot]{glossaries}         % glossaire / lexisque
\usepackage{lastpage}
\usepackage{tabularx}
\def\arraystretch{1.2}
\usepackage{multirow}
\usepackage{float}
\usepackage{enumitem}
\usepackage{pdfpages}
\usepackage{caption}						% sub figures
\usepackage{subfig}
\usepackage{subcaption}
\usepackage{url}							% gestion des URL
%\usepackage{minted}						% code sytax highlight
\usepackage{pslatex}    					% utiliser des fontes de caractères postscript pour la compatibilité	
%\usepackage[a-1b]{pdfx}					% test de compatibilité a1b - il faut aussi ajoute les fichiers
\usepackage{multicol}						% multiple columns
\setlength{\columnsep}{0.8cm}				% multicol column separation space
\usepackage{longtable}                      % table on multiple pages
\setlength{\LTpost}{10pt}                   % space afer lontable
\pdfminorversion=7  						% version 1.7

\geometry{hmargin=2.5cm,vmargin=3.0cm}
\usepackage[ddmmyyyy]{datetime}
\renewcommand{\dateseparator}{.}

\setcounter{secnumdepth}{4}
\setcounter{tocdepth}{2}					% profondeur table des matières

% Titre et auteur 
\title{{LPSC - Julia set report}}
%\author{Alain Chappuis}
\author{Adrien Balleyguier}
\date{\today}

\usepackage{color}
\definecolor{deepblue}{rgb}{0,0,0.5}
\definecolor{deepred}{rgb}{0.6,0,0}
\definecolor{deepgreen}{rgb}{0,0.5,0}
\usepackage{listings}
\lstdefinestyle{numbers}{numbers=left, stepnumber=5, numberstyle=\tiny, numbersep=10pt}
\lstdefinestyle{font}{
    basicstyle=\ttfamily,%\scriptsize,
    columns=fullflexible,
    keepspaces=true,
    showstringspaces=false,
    extendedchars=true,
    upquote=true,
    breaklines=true,
    showtabs=false,
    showspaces=false,
    showstringspaces=false,
    keywordstyle=\color[rgb]{0,0,1},
    commentstyle=\color[rgb]{0.43, 0.5, 0.5},
    stringstyle=\color[rgb]{0.0,0.5,0.0}
}
\lstdefinestyle{frame}{frame=lines}
\lstdefinestyle{caption}{numberbychapter=false,captionpos=b,caption=\lstname}
\lstdefinestyle{CStyle}{language=C,style=numbers,style=caption,style=font,style=frame}
\lstset{language=C}
%using 'inform' as nameholder for console output style
\lstdefinestyle{outputStyle}{language=inform,style=frame,style=caption,basicstyle=\ttfamily,keywordstyle=\ttfamily}
\lstset{language=inform}

\setlength{\doublerulesep}{0pt}
\newcommand{\thickhline}{\hline\hline\hline}

\makeatletter
\let\thetitle\@title
\let\theauthor\@author 
\let\thedate\@date
\usepackage[
            pdfauthor={\theauthor},
            pdftitle={\thetitle},
            pdfproducer={Latex with hyperref},
            pdfsubject={\thetitle},
            pdfkeywords={BLE, NFC, encryption, pairing, OOB, ECC},
            pdfcreator={pdflatex, or other tool}]
            {hyperref}               		% gestion des hyperliens
            \hypersetup{
                colorlinks,
                citecolor=black,
                filecolor=black,
                linkcolor=black,
                urlcolor=deepblue
            }
            \hypersetup{pdfstartview=XYZ}   		% zoom par défaut
% glossary guimauve

\renewcommand\paragraph{\@startsection{paragraph}{4}{\z@}%
                                    {3.25ex \@plus1ex \@minus.2ex}%
                                    {-1em}%
                                    {\normalfont\normalsize\bfseries}}

\newglossarystyle{modsuper}{
    \glossarystyle{super}
    \renewcommand{\glsgroupskip}{}
}
\setlength{\glsdescwidth}{0.8\hsize}
% remove glossary original title
\renewcommand{\glossarysection}[2][]{} 

% list of figures / tables as subsection
\renewcommand\listoffigures{%
	\subsection{{Liste of figures}}%
	\@mkboth{\MakeUppercase\listfigurename}{\MakeUppercase\listfigurename}%
	\@starttoc{lof}%
}
\renewcommand\listoftables{%
	\subsection{{Liste of tables}}%
	\@mkboth{\MakeUppercase\listtablename}{\MakeUppercase\listtablename}%
	\@starttoc{lot}%
}


% item list
%\renewcommand{\labelitemi}{$\bullet$}
\renewcommand{\labelitemii}{\small$\circ$}
%\renewcommand{\labelitemii}{$\bullet$}
%\renewcommand{\labelitemiv}{$\bullet$}
\gdef\lst@numberfirstlinefalse{\global\let\lst@ifnumberfirstline\iffalse}
\lst@AddToHook{Init}{\global\let\lst@ifnumberfirstline\iftrue}
\makeglossaries

% Footer and header
\pagestyle{fancy}
\fancyhf{}

%\rhead{\theauthor \\ T3-a}
%\lhead{\thetitle \\ Projet de semestre 5 }
\renewcommand{\sectionmark}[1]{\markboth{#1}{}}
\rhead{\theauthor\\ \thedate}      % \: pour gros espace
\lhead{\thetitle \\ \nouppercase{\leftmark} }


% Refer to a previously written footnote is to define
\newcommand{\footnoteref}[1]{\textsuperscript{\ref{#1}}}

%-------------------------------------------------------------------------------------------------------
